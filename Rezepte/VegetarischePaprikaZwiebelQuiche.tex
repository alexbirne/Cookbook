\section*{Vegetarische Paprika Zwiebel Quiche}
\ihead{}\chead{Personen:6}\ohead{}
\ifoot{Vorbereitungszeit:\SI{30}{Min}}\cfoot{}\ofoot{Kochzeit: \SI{120}{Min}}
{\Large Zutaten}
\begin{multicols}{2}
\begin{itemize}
    \item \SI{200}{g} Mehl
    \item \SI{100}{g} Butter
    \item \num{3} Eier
    \item \num{1} Schuss Mineralwasser
    \item \num{1} Prise Salz
    \item \num{3} rote Paprika
    \item \num{3} kleine Zwiebeln
    \item \SI{120}{g} Ajvar
    \item \SI{100}{g} Creme fraiche
    \item \num{1} EL Tomatenmark
    \item \SI{200}{g} geriebener Käse
    \item Salz, Pfeffer
\end{itemize}
\end{multicols}
\noindent
{\Large Zubereitung}\\
\\
\textit{Vorbereitung:} Ofen auf \SI{190}{\celsius} Ober-/Unterhitze vorheizen.\\
Aus Mehl, Butter, Ei, Mineralwasser und etwas Salz einen Mürbeteig kneten. 
Danach den Teig in Klarsichtfolie einrollen und im Kühlschrank etwa \SI{1}{h} ziehen lassen.
Jetzt die Paprika und Zwiebeln fein würfeln. 
Das Gemüse in einer Pfanne bei geschlossenem Deckel so lange anschwitzen bis es glasig ist. 
Das dauer etwa \SI{20}{min}. 
Etwas abkühlen lassen. 
Den Mürbeteig ausrollen und eine gefette 26er Springform damit auskleiden. 
Den Boden mit einer Gabel mehrfach einstechen. 
Gemüse, \num{2} Eier, Käse, Ajvar, Tomatenmark und Creme fraiche miteinander vermischen und mit Salz und Pfeffer abschmecken.
Die Masse gleichmäßg in der Springform verstreichen.
Jetzt die Quiche \SI{40}{min} backen.
Etwas abkühlen lassen vor dem Servieren. 
