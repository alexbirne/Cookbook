\section*{Gefüllte Tomaten auf griechische Art}
\ihead{}\chead{Personen: 4}\ohead{}
\ifoot{Vorbereitungszeit:\SI{25}{Min}}\cfoot{}\ofoot{Kochzeit: \SI{30}{Min}}
{\Large Zutaten}
\begin{multicols}{2}
\begin{itemize}
    \item \num{8} Fleischtomaten
    \item \SI{500}{g} gemischtes Hackfleisch
    \item \SI{300}{g} Kritharaki
    \item \SI{400}{g} Fetakäse
    \item \num{6} TL Gyrosgewürz
    \item \SI{200}{ml} Gemüsebrühe
    \item Salz, Pfeffer, Öl
\end{itemize}
\end{multicols}
\noindent
{\Large Zubereitung}\\
\\
\textit{Vorbereitung:} Ofen auf \SI{180}{\celsius} Umluft vorheizen.\\
Die Fleischtomaten waschen und am Stielansatz kegelförmig ausschneiden sodass eine ausreichend große Öffnung zum Füllen entsteht.
Das Fruchtfleisch entfernen. 
Daraus lässt sich eine Sauce kochen. 
Die Kritharaki knapp bissfest kochen. 
Das Hackfleisch knusprig braten und mit dem Gyrosgewürz, Salz und Pfeffer abschmecken.
Die Nudeln dazugeben. 
Den Fetakäse klein würfeln und vorsichtig unterheben. 
Alles gut verrühren. 
Die Tomaten reichlich mit der Masse füllen und in eine Auflaufform setzen. 
Die Gemüsebrühe angießen, sodass der Boden bedeckt ist. 
\SI{20}{25}{min} backen. 
