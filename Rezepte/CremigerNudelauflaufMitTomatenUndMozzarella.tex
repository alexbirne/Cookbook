\section*{Cremiger Nudelauflauf mit Tomaten und Mozzarella}
\ihead{}\chead{Personen: 4}\ohead{}
\ifoot{Vorbereitungszeit:30 Min}\cfoot{}\ofoot{Kochzeit:30 Min}
{\Large Zutaten}
\begin{multicols}{2}
\begin{itemize}
    \item \SI{500}{g} Rigatoni
    \item \num{1} Zwiebel
    \item \num{2} Knoblauchzehen
    \item \num{1} Chilischote
    \item \num{1} Becher Sahne
    \item \SI{50}{g} Parmesan, gerieben
    \item \SI{125}{g} Mozzarella
    \item \SI{400}{g} Cherrytomaten
    \item \num{1} Bund Basilikum
    \item Zucker, Salz, Pfeffer, Olivenöl, Zucker
\end{itemize}
\end{multicols}
\noindent
{\Large Zubereitung}\\
\\
\textit{Vorbereitung:} Ofen auf \SI{180}{\celsius} Umluft vorheizen.\\
Die Zwiebel und den Knoblauch sehr fein schneiden.
Die Chilischote entkernen und ebenso fein hacken.
Die Kirschtomaten waschen und halbieren.
Den Parmesan reiben und den Mozzarella grob würfeln.
Die Basilikumblätter abzupfen, waschen und trocken tupfen.
In einem großen Topf Salzwasser zum Kochen bringen und die Nudeln darin laut Packungsangabe al dente garen.
Die passierten Tomaten hinzufügen und die Sauce ein paar Minuten leicht köcheln lassen.
Dann die Sahne und den geriebenen Parmesan unterrühren und die Sauce mit Salz, Pfeffer und einer ordentlichen Prise Zucker abschmecken. 
Wenn die Nudeln soweit sind, diese abgießen und in die Pfanne zur Sauce geben.
Die Pfanne von der Hitze nehmen und die halbierten Kirschtomaten und die Hälfte der Mozzarellawürfel unterheben.
Die Basilikumblätter in Streifen schneiden und ebenfalls unterheben.
Alles zusammen in eine Auflaufform geben, mit dem restlichen Mozzarella bestreuen und ca. 20 Minuten auf der mittleren Schiene im Backofen gratinieren.
