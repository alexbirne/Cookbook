\section*{Kritharaki mit gerösteter Spitzpaprika}
\ihead{}\chead{Personen:}\ohead{}
\ifoot{Vorbereitungszeit:}\cfoot{}\ofoot{Kochzeit:\SI{45}{Min}}
{\Large Zutaten}
\begin{multicols}{2}
\begin{itemize}
    \item \SI{1}{Pkg} Zatziki
    \item \SI{180}{g} Kritharaki
    \item \SI{4}{g} Gemüsebrühe
    \item \SI{125}{g} Kirschtomaten
    \item \SI{70}{g} Oliven ohne Stein
    \item \SI{5}{g} Petersilie
    \item \num{2} rote Spitzpaprika
    \item \num{1} Zucchini
    \item \num{2} Knoblauchzehen
    \item Öl, Weißweinessig
    \item Salz, Pfeffer, Paprika
\end{itemize}
\end{multicols}
\noindent
{\Large Zubereitung}\\
\\
\textit{Vorbereitung:} Ofen auf \SI{200}{\celsius} Umluft vorheizen.\\
Spitzpaprika längs halbieren und Kerngehäuse entfernen.
Zucchini längs halbieren und in \SI{1}{cm} dicke Halbmonde schneiden.
Spitzpaprika mit der Öffnung nach unten, Zucchine und Kirschtomaten nebeneinander auf ein Backblech legen und mit \num{1} EL Öl, Salz und Pfeffer mischen.
Alles für \SIrange{20}{25}{min} backen.
Knoblauch abziehen und für \SI{10}{min} mit in den Ofen geben.
In einem Topf Kritharaki in \num{1} Öl für \SIrange{2}{3}{min} anschwitzen. Mit \SI{400}{ml} ablöschen und mit Gemüsebrühpulver, Paprikagewürz und Salz würzen. 
Einmal aufkochen lassen und dann für \SIrange{10}{12}{min} bei niedriger Hitze köcheln lassen.
Danach weitere \SI{10}{min} stehen lassen und quellen lassen. 
Petersilienblätter fein hacken, Oliven halbieren und mit \num{0,5} EL Olivenöl und der Zucchini unter die fertigen Nudeln mischen und mit Salz und Pfeffer abmschmecken. 
Kritharaki auf Tellern verteilen, Spitzpaprika und Kirschtomaten darauflegen und mit Zatziki genießen. 
