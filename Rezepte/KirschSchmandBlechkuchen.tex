\section*{Kirsch-Schmand-Blechkuchen}
\ihead{}\chead{Personen:}\ohead{}
\ifoot{Vorbereitungszeit:}\cfoot{}\ofoot{Kochzeit:}
{\Large Zutaten}
\begin{multicols}{2}
\textit{Boden:}
\begin{itemize}
    \item \SI{250}{g} Butter
    \item \SI{200}{g} Mehl
    \item \SI{200}{g} Zucker
    \item \num{4} Eier
    \item \num{2} TL Backpulver
    \item \num{2} Pkg. Vanillezucker
\end{itemize}
% \columnbreak
\textit{Füllung:}
\begin{itemize}
    \item \num{2} Pkg. Puddingpulver, Vanille
    \item \num{2} Becher Schmand
    \item \SI{1}{l} Milch
    \item \num{2} Eigelb
    \item \num{2} Gläser Sauerkirschen
\end{itemize}
\textit{Belag:} \num{2} Pkg Tortenguss
\end{multicols}
\noindent
{\Large Zubereitung}\\
\\
\textit{Vorbereitung:} Ofen auf \SI{180}{\celsius} Ober-/Unterhitze vorheizen.\\
Zutaten für den Boden verrühren und auf ein Blech streichen, das mit Backpapier ausgelegt ist.
Etwa \SI{12}{min} backen.
Pudding kochen, wenn er abgekühlt ist die Eigelb und den Schmand unterrühren. 
Auf dem Boden verteilen, Kirschen obenauf.
Nochmals \SI{15}{min} backen. 
Guss aus Kirschsaft herstellen und auf dem Kuchen verteilen. 
