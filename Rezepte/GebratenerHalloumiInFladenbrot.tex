\section*{Gebratener Halloumi in Fladenbrot}
\ihead{}\chead{Personen:2}\ohead{}
\ifoot{Vorbereitungszeit:}\cfoot{}\ofoot{Kochzeit:\SIrange{30}{40}{min}}
{\Large Zutaten}
\begin{multicols}{2}
\begin{itemize}
    \item \num{2} Minifladenbrote
    \item \num{1} Salatherz
    \item \SI{75}{g} Joghurt
    \item \num{1} Knoblauchzehe
    \item \num{1} Zwiebel
    \item \SI{25}{g} Ajvar
    \item \num{1} Gurke
    \item \num{1} Tomate
    \item \num{1} Zucchini
    \item \SI{10}{g} Dill \& Petersilie
    \item \SI{200}{g} Halloumi
    \item \SI{17}{ml} Mayonnaise
    \item Salz, Pfeffer, Olivenöl, Gewürze
\end{itemize}
\end{multicols}
\noindent
{\Large Zubereitung}\\
\\
\textit{Vorbereitung:} Ofen auf \SI{200}{\celsius} Umluft vorheizen.\\
Für das Ofengemüse: Zucchini und Zweibel in etwa \SI{1}{cm} dünne Scheiben schneiden.
Auf einem Backblech verteilen.
Mit einem EL Öl und etwas Salz und Pfeffer vermengen und etwa \SI{20}{min} garen.

Joghurtdip vorbereiten: Die Hälfte des Knoblauchs in eine kleine Schüssel pressen.
Dill und Petersilie fein hacken. 
Kräuter mit Mayonnaise und der Hälfte des Joghurts in der Schüssel glatt rühren.
Mit Salz und Pfeffer abschmecken.

Ajvardip vorbereiten: Restlichen Knoblauch in eine kleine Schüssel pressen. 
Ajvar hinzugeben und vermengen. 
Mit Joghurt verrühren.
Mit Salz und Pfeffer abschmecken.

Gemüse schneiden: Tomate und Gurke in dünne Scheiben schneiden. 
In den letzten \SI{1}{0min} das Fladenbrot im Backofen aufwärmen.
Halloumi in \num{6} Scheiben schneiden. 
Mit einem EL Öl und Gewürzen marinieren. 

Halloumi braten: 
In einer Pfanne \num{1} EL Öl erhitzen und Halloumi von beiden Seiten \SIrange{3}{4}{min} braten. 
Blätter vom Salat abzupfen. 

Fladenbrot füllen: 
Fladenbrot mit den beiden Dips bestreichen. 
Mit Salat, Tomate und Gurke belegen. 
Halloumi und das Ofengemüse dazugeben. 