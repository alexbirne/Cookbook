\section*{Paprika-Pastasotto}
\ihead{}\chead{Personen: 2}\ohead{}
\ifoot{Vorbereitungszeit:10 Min}\cfoot{}\ofoot{Kochzeit:20 Min}
{\Large Zutaten}
\begin{multicols}{2}
\begin{itemize}
    \item \SI{150}{g} Griechische Reisnudeln (Kritharaki)
    \item \num{1} EL Olivenöl
    \item \num{1} Zwiebel
    \item \num{1} Knoblauchzehe
     \item \num{2} kleine oder eine große Paprikaschote, rot
    \item \SI{500}{ml} Hühnerbrühe
    \item \num{1} TL Paprikapulver, edelsüß
    \item \num{1} EL Schmand
    \item Salz
\end{itemize}
\end{multicols}
\noindent
{\Large Zubereitung}\\
\\
Die Zwiebel und den Knoblauch schälen und in kleine Würfel schneiden.
Die Paprikaschoten waschen und in kleine Würfel schneiden.
Das Olivenöl in einem Topf erhitzen und darin die Zwiebelwürfel glasig anschwitzen, Knoblauch zufügen und kurz mit anschwitzen lassen.
Nun die Reisnudeln in den Topf geben und ebenfalls kurz anschwitzen lassen.
Die Paprikawürfel zufügen und mit etwas Hühnerbrühe ablöschen.
Nun nach und nach die Brühe angießen.
Nach 20 Minuten sind die Nudeln gar, nun noch mit Paprika und Salz kräftig abschmecken und einen Esslöffel Schmand unter das Pastasotto rühren.
