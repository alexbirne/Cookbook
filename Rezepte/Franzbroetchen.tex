\section*{Franzbrötchen}
\ihead{}\chead{Personen:}\ohead{}
\ifoot{Vorbereitungszeit:\SI{25}{Min}}\cfoot{}\ofoot{Kochzeit:\SI{20}{Min}}
{\Large Zutaten}
\begin{multicols}{2}
\textit{Für den Teig:}
\begin{itemize}
    \item \SI{500}{g} Mehl
    \item \SI{250}{ml} Milch
    \item \SI{70}{g} Butter
    \item \num{1} Pkg Vanillezucker
    \item \num{1} Pkg Trockenhefe
    \item \num{1} Ei
    \item \num{1} Prise Salz
\end{itemize}
\textit{Für die Füllung:}
\begin{itemize}
    \item \SI{150}{g} Butter
    \item \SI{200}{g} Zucker
    \item \num{5} TL Zimt
\end{itemize}
\end{multicols}
\noindent
{\Large Zubereitung}\\
\\
\textit{Vorbereitung:} Ofen auf \SI{50}{\celsius} Umluft vorheizen.\\
Milch in einem Topf erwärmen und die Butter darin schmelze lassen.
Das Gemisch abkühlen lassen. 
Mehl mit der Hefe vermischen und nach und nach die anderen Zutaten inklusive der lauwarmen Butter-Milch-Mischung unterheben und mit einem Mixer etwa \SI{5}{Min} zu einem glatten Teig verkneten.
Den Teig an einem warmen Ort zugedeckt gehen lassen bis sich das VOlumen verdoppelt hat. 
Am Besten geht das im Backofen bei \SI{50}{\celsius} innerhalb von \SIrange{30}{40}{Min}. 
In der Zwischenzeit für die Füllung die Butter schmelzen und gut mit Zucker und Zimt vermischen. 
Die Teigschüssel aus dem Backofen holen und den Backofen auf \SI{160}{\celsius} Umluft vorheizen.
Zwei Backbleche mit Backpapier auslegen. 
Den Teig auf einer bemehlten Fläche zu einem \num{25}x\SI{70}{cm} Rechteck ausrollen. 
Die Füllung gleichmäßig darauf verteilen sodass noch etwas übrig bleibt. 
Das Rechteck einrollen.
Nun etwa zwölf Trapeze schneiden. 
Das Trapez mit der breiteren Fläche auf das Backpapier setzen und mit Hilfe eines Kochlöffels eine Mulde bis zur Arbeitsplatte drücken.
Die Franzbrötchen noch einmal etwas mit der Füllung bestreichen und \SIrange{15}{20}{Min} backen. 
