\section*{Paprika-Hack-Pfanne mit Feta}
\ihead{}\chead{Personen: 3}\ohead{}
\ifoot{Vorbereitungszeit: \SI{20}{Min}}\cfoot{}\ofoot{Kochzeit: \SI{25}{Min}}
{\Large Zutaten}
\begin{multicols}{2}
\begin{itemize}
    \item \SI{500}{g} gemischtes Hackfleisch
    \item \num{0.5} Tube Tomatenmark
    \item \num{1} Bund Frühlingszwiebeln
    \item \num{3} Paprika
    \item \num{1} Zwiebel
    \item Kräuter nach Wahl
    \item \SI{240}{ml} Brühe
    \item \num{1} Becher Sahne
    \item \num{2} Ecken Sahneschmelzkäse
    \item \num{1} Pkg.Fetakäse
    \item Salz, Pfeffer, Parika
\end{itemize}
\end{multicols}
\noindent
{\Large Zubereitung}\\
\\
Hackfleisch anbraten und kräftig würzen.
Dann die kleingeschnittenen Paprika, Fühlingszwiebeln, Zwiebeln und Kräuter dazugeben. 
\SI{5}{min} durchschmoren lassen.
Dann das Tomatenmark dazugeben und mit der Brühe ablöschen.
Bei kleiner Flamme \SI{5}{min} köcheln lassen. 
Jetzt die Sahne und den Schmelzkäse zugeben und nochmals \SIrange{5}{10}{min} köcheln lassen. 
Mit Salz und Pfeffer abschmecken und von der Kochstelle nehmen.
Feta in Würfel schneiden und auf die Parika-Hack-Pfanne geben.
