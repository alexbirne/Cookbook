\section*{Crespelle a la Bolognese}
\ihead{}\chead{Personen: 4}\ohead{}
\ifoot{Vorbereitungszeit:}\cfoot{}\ofoot{Kochzeit:}
{\Large Zutaten}
\begin{multicols}{2}
\textit{Crespelle:}
\begin{itemize}
    \item \SI{60}{g} Mehl
    \item \num{1} Prise Salz
    \item \SI{125}{ml} Milch
    \item \num{4} Eier
    \item \SI{30}{g} Butter
\end{itemize}
\textit{Bolognese:}
\begin{itemize}
    \item \num{1} Zwiebel, gewürfelt
    \item \SI{20}{g} Butter
    \item \num{1} Knoblauchzehe
    \item \SI{400}{g} Hackfleisch, gemsicht
    \item \num{1} kleine Dose geschälte Tomaten
\end{itemize}
\textit{Bechamel:}
\begin{itemize}
    \item \SI{40}{g} Butter 
    \item \SI{40}{g} Mehl
    \item \SI{500}{ml} Milch
    \item Parmesan
    \item Salz, PFeffer, Muskat
\end{itemize}
\end{multicols}
\noindent
{\Large Zubereitung}\\
\\
\textit{Vorbereitung:} Ofen auf \SI{170}{\celsius} Umluft vorheizen.\\
Für die Crespelle den Teig zusammenrühren und portionsweise in einer Pfanne ausbacken.
Zwiebeln und Knoblauch glasig dünsten.
Hackfleisch dazugeben und mit den Tomaten auffüllen.
Mit Salz und PFeffer abschmecken und bei kleiner Hitze \SI{20}{min} garen.
Für die Bechamelsauce \SI{40}{g} Butter erhitzen und \SI{40}{g} Mehl darüber stäuben.
Milch dazugeben und bei kleiner Hitze \SI{10}{min} einkochen lassen. 
Mit Salzn und Muskat abschmecken.
Crespelle mit Hackfleisch bestreichen aufrollen und nebeneinander in eine Form legen. 
Die Bechamelsauce darüber geben und den Parmesan darauf verstreuen.
Bei \SI{170}{\celsius} im vorgeheizten Ofen etwa \SIrange{15}{20}{min} backen.