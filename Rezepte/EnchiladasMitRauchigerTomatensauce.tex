\section*{Enchiladas mit rauchiger Tomatensauce}
\ihead{}\chead{Personen:}\ohead{}
\ifoot{Vorbereitungszeit:}\cfoot{}\ofoot{Kochzeit:\SI{40}{Min}}
{\Large Zutaten}
\begin{multicols}{2}
\begin{itemize}
    \item \SI{50}{g} Gouda
    \item \SI{150}{g} Champignons
    \item \SI{70}{g} Tomatenmark
    \item \SI{150}{g} Schmand
    \item \SI{4}{g} Gemüsebrühe
    \item \num{4} Tortilla Wraps
    \item \num{1} Dose schwarze Bohnen
    \item \num{1} gelbe Paprika
    \item \num{1} Limette
    \item \num{1} Frühlingszwiebel
    \item \num{1} rote Chilischote
    \item \num{2} Tomaten
    \item \num{1} Knoblauchzehe
    \item Öl, Wasser, Zucker, Butter
    \item Salz, Pfeffer, Gewürze
\end{itemize}
\end{multicols}
\noindent
{\Large Zubereitung}\\
\\
\textit{Vorbereitung:} Ofen auf \SI{180}{\celsius} Umluft vorheizen.\\
Paprika halbieren, entkernen und in Streifen schneiden.
Schwarze Bohnen unter fließendem Wasser abspülen, Knoblauch abziehen und fein hacken.
Chili in Ringe schneiden, Champignons je nach Größe viertel oder achteln.
Weißen und grünen Teil der Frühlingszwiebel getrennt voneinander in dünne Ringe schneiden.
Anschließend in einer großen Pfanne Knoblauch, Paprika, Champignons und weiße Frühlingszwiebelringe in \num{1} EL Öl anschwitzen.
Nach \SI{3}{min} schwarze Bohnen hinzugeben und für weitere \SI{2}{min} anbraten.
\SI{100}{ml} Wassser und Gemüsebrühe hinzugeben und solange einköcheln lassen bis die Paprika weich ist und das Wasser verkocht ist.
Nach Ende der Garzeit \num{2} EL Schmand in die Pfanne geben, nach Geschmack mit Salz und Pfeffer würzen.
Die Tortillas mit der Bohnenmischung füllen, zusammenrollen und in eine Auflaufform legen.
In derselben Pfanne \num{1} TL Butter schmelzen, nach Geschmack würzen (mexikanisch/Steak) und kurz erhitzen. 
Tomatenmark und \SI{150}{ml} Wasser dazugeben und zu einer Sauce verrühren. 
Mit Salz und Pfeffer abschmecken, über den Enchiladas verteilen und mit Chiliringen toppen.
Gouda darüber bröseln und für \SIrange{10}{15}{min} im Ofen backen. 
Für die Tomatensalsa die Limette heiß abwaschen, halbieren und die Hälfte in einer Schüssel auspressen.
Die Tomaten in etwa \SI{2}{cm} große Würfel schneiden und mit dem Limettensaft, den grünen Frühlingszwiebelringen und \num{1} TL Zucker vermengen.
Für den Limettendip den restlichen Schmand mit \num{1} TL Limettenschal verrühren und mit Salz und Pfeffer abschmecken. 
