\section*{Apfelkuchen, gedeckt}
\ihead{}\chead{Personen:}\ohead{}
\ifoot{Vorbereitungszeit:}\cfoot{}\ofoot{Kochzeit:}
{\Large Zutaten}
\begin{multicols}{2}
\textit{Teig:}
\begin{itemize}
    \item \SI{300}{g} Mehl
    \item \num{2} TL Backpulver
    \item \SI{100}{g} Zucker
    \item \num{1} Pkg Vanillezucker
    \item Salz
    \item \num{1} Ei
    \item \num{1} EL Milch
    \item \SI{150}{g} Butter
\end{itemize}
\textit{Apfelmasse:}
\begin{itemize}
    \item \SI{1}{kg} Äpfel
    \item \SI{100}{g} Zucker
    \item \num{1} TL Zimt
    \item \SI{50}{g} Rosinen
    \item \num{1} EL Rum
    \item Puderzucker, Zitronensaft
\end{itemize}
\end{multicols}
\noindent
{\Large Zubereitung}\\
\\
\textit{Vorbereitung:} Ofen auf \SI{200}{\celsius} Ober-/Unterhitze vorheizen.\\
Aus Mehl, Backpulver, Zucker, Vanillezucker, Salz, Ei, Milch und Butter einen Knetteig herstellen, \SI{30}{min} kaltstellen.
In zwei Teilen ausrollen und einen davon in eine gefettete Auflaufform geben.
Äpfel schälen, in Stücke schneiden. 
Dann \SI{50}{g} Zucker, \num{1/2} TL Zimt und \SI{50}{g} Rosinen mit den Äpfeln andünsten.
\num{1} EL Rum darübergeben, vermengen, und auf den Boden geben.
Den Teigdeckel darauf geben und bei \SI{200}{\celsius} bei Ober-/Unterhitze \SI{30}{min} backen.
Puderzucker und Zitronensaft vermischen und nach dem Erkalten mit Glasur ebstreichen. 