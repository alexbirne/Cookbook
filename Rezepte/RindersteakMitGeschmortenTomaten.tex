\section*{Rindersteak mit geschmorten Tomaten einem frischen Salat und Kartoffelecken}
\ihead{}\chead{Personen:2}\ohead{}
\ifoot{Vorbereitungszeit:}\cfoot{}\ofoot{Kochzeit:\SI{15}{min}}
{\Large Zutaten}
\begin{multicols}{2}
\begin{itemize}
    \item \SI{250}{g} Rinderhüftsteak
    \item \SI{20}{g} Aioli
    \item \SI{400}{g} vorgegarte Kartoffelwürfel
    \item \SI{10}{g} glatte Petersilie/Thymian
    \item \SI{20}{g} Butter
    \item \SI{250}{g} Kirschtomaten
    \item \SI{15}{ml} Basilikumpaste
    \item \SI{50}{g} Salatmischung
    \item Salz, Pfeffer, Olivenöl, Essig
\end{itemize}
\end{multicols}
\noindent
{\Large Zubereitung}\\
Dressing \& Kartoffelwürfel: In einer großen Schüssel Basilikumpaste zusammen mit \num{1} EL Essig verrühren und mit Salz und Pfeffer abschmecken.
Petersilienblätter abzupfen und ggf. fein hacken.
Butter in einer großen Pfanne erhitzen.
Kartoffelwürfel in der Pfanne verteilen und etwa \SIrange{5}{8}{min} unter gelegentlichem Wenden erhitzen. 
Danach Petersilienblätter unterheben.
Währenddessen mit dem Rezept fortfahren. 

Steak braten: Rindersteak von beiden Seiten salzen.
In einer großen Pfanne \num{1} EL Öl erhitzen, Rindersteaks, Thymianzweige, und Kirschtomaten zugeben und Steaks darin auf jeder Seite \SIrange{5}{6}{min} anbraten (medium). 
Die Kirschtomaten sollten danach leicht aufgeplatzt und etwas weicher geworden sein.
Steak aus der Pfanne nehmen, kurz ruhen lassen und portionieren.
Kirschtomaten in der Pfanne lassen, damit sie warm bleiben. 

Anrichten: Salat zum Dressing in dei Schüssel geben und vorsichtig marinieren. 
Kartoffelwürfel, Rindersteaks, Tomaten und Salat auf Tellern anrichten., Aioli dazureichen und genießen. 