\section*{Rucola-Risotto mit Schinken}
\ihead{}\chead{Personen: 4}\ohead{}
\ifoot{Vorbereitungszeit:}\cfoot{}\ofoot{Kochzeit:}
{\Large Zutaten}
\begin{multicols}{2}
\begin{itemize}
    \item \SI{400}{g} Risottoreis
    \item \num{2} Bund Rucola
    \item \SI{150}{g} Parmaschinken
    \item \num{3} Knoblauchzehen
    \item \num{1} Stange Lauch
    \item \num{4} EL Butter
    \item \num{2} Tomaten
    \item \SI{1}{l} Gemüsebrühe
    \item \num{2} EL Creme fraiche
    \item \SI{50}{g} Parmesan
    \item Salz, Pfeffer, Muskat
\end{itemize}
\end{multicols}
\noindent
{\Large Zubereitung}\\
\\
Tomaten enthäuten und mit Knoblauch, Lauch, Rucola und Schinken klein schneiden.
\num{2} EL Butter erhitzen, den Knoblauch und Lauch darin andünsten.
Den Reis und \num{3/4} des Rucola hinzugeben udn weiter dünsten.
Etwa \num{2} Tassen der Brühe angießen und einkochen lassen. 
Dann nach und nach die restliche Brühe angießen und bei schwacher Hitze den Reis quellen lassen, bis er al dente ist.
Kurz vor Ende der Garzeit die Tomaten, den Schinken, Parmesan, Butter und Creme fraiche unterrühren und abschmecken. Mit restlcihem Rucola servieren.