\section*{Risotto Tricolore}
\ihead{}\chead{Personen: 3}\ohead{}
\ifoot{Vorbereitungszeit:\SI{1}{h}}\cfoot{}\ofoot{Kochzeit:\SI{30}{min}}
{\Large Zutaten}
\begin{multicols}{2}
\begin{itemize}
    \item \num{1} große Zwiebel, fein gewürfelt
    \item \num{1} kleine Zucchini, gewürfelt
    \item Maggi
    \item \num{4} EL Olivenöl
    \item \num{1} EL Zucker
    \item \num{4} Tomaten
    \item \num{2} Knoblauchzehen
    \item \num{1} TL Oregano
    \item \SI{200}{g} Reis
    \item \SI{250}{ml} trockener Weißwein
    \item \SI{500}{ml} Gemüsebrühe
    \item Salz, Pfeffer
\end{itemize}
\end{multicols}
\noindent
{\Large Zubereitung}\\
\\
Die Zwiebeln ganz fein würfeln, die Zucchinie ebenfalls und beides zusammen mit wenigen Spritzern Maggi, \num{2} EL Olivenöl und Zucker in eine Schüssel geben und vermischen.
Die Schüssel abgedeckt beiseite stellen und die \textbf{Mischung ca. \SI{1}{h} durchziehen lassen}.
Die Tomaten waschen und abtrocknen, halbieren, den Strunk und die Kerne heraus schneiden, sodass nur noch das Tomatenfleisch übrig bleibt, anschließend in \SI{1}{cm} Würfel schneiden und in eine Schüssel geben. 
Den Knoblauch dazu pressen, dann \num{2} EL Olivenöl, Oregano, Salz und PFeffer dazugeben und wieder alles durchmischen, abdecken und ebenfalls \textbf{ca. \SI{1}{h} durchziehen lassen}.
Danach eine große und tiefe Pfanne (oder Topf) stark erhitzen.
Wenn die Pfanne heiß ist, die ZUcchini-Zwiebel-Mischung hinein geben und ohne Deckel leicht anbräunen.
Dann die Tomaten-Mischung mit in die Pfanne geben und ca. \SI{3}{min} unter Rühren mit anbraten.
Wichtig: Nach \SI{1}{min} den Deckel auf die Pfanne geben, damit sich schon ein bisschen Flüssigkeit sammelt und nicht verdampft.
Das angebratene Gemüse komplett am Pfannenboden verteilen, dann den Reis gleichmäßig darauf verteilen, den Decken auflegen und jetzt die Hitze auf ca. \num{2/3} der Leistung reduzieren.
Alles kurz köcheln lassen, der Reis saugt dabei die Flüssigkeit auf, dann den Deckel abnehmen, Reis und Gemüse gut durchmischen und den Weißwein dazugeben.
Nochmals alles gut druchrühren und aufkochen lassen.
Den Deckel jetzt nicht mehr auf die Pfanne setzen.
Wichitg: Den Reis nicht anbrennen lassen, immer wieder gut umrühren.
Wenn in der Pfanne die Flüssigkeit verschwindet, mit Brühe nachfüllen, immer so viel, dass der Reis nicht ganz bedeckt ist und weiter rühren.
Wenn nach mehrmaligem Nachfüllen von Brühe der Reis eine sämige Konsistenz hat, ist das Risotto fertig.