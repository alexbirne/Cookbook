\section*{Paprika-Tomaten-Cremesuppe mit würzigem Hackfleisch}
\ihead{}\chead{Personen: 8}\ohead{}
\ifoot{Vorbereitungszeit:}\cfoot{}\ofoot{Kochzeit: 25 min}
{\Large Zutaten}
\begin{multicols}{2}
\begin{itemize}
    \item \num{2} große Zwiebeln
    \item \num{2} Knoblauchzehen
    \item \num{4} sehr große rote Paprikaschoten
    \item \num{4} EL Rapsöl
    \item \SI{500}{g} Hackfleisch gemischt 
    \item \num{4} TL Paprika rosenscharf
    \item \num{3} EL Weizenmehl
    \item \SI{1.4}{l} klare Gemüsebrühe
    \item \num{8} EL Tomatenmark
    \item \SI{200}{g} Schlagsahne
    \item Salz, Pfeffer
\end{itemize}
% \columnbreak
\end{multicols}
\noindent
{\Large Zubereitung}\\
\\
Zwiebeln und Knoblauch schälen und fein würfeln bzw. hacken.
Paprika waschen, halbieren, Kerne entfernen und das Fruchtfleisch würfeln.
\num{2} EL Öl in einer Pfanne erhitzen, Hackfleisch darin rundherum braun und krümelig anbraten.
Mit Salz, Pfeffer und \num{2} TL Paprika rosenscharf würzen.
Restliches Öl (\num{2} EL) in einem Topf erhitzen, Zwiebel- und Knoblauchwürfel darin andünsten. 
Paprikawürfel hinzugeben und mit Salz, Pfeffer und \num{2} TL Rosenscharf würzen.
Mit Mehl bestäuben, kurz anschwitzen.
Gemüsebrühe und Tomatenmark hinzugeben, aufkochen lassen und ca. \SI{10}{min} kochen lassen.
Suppe mit dem Pürierstab pürieren, ggf. nachwürzen und die Hälfte der Sahne unterrühren.
Auf Teller verteilen, Hackfleisch und restliche Sahne hinzugeben und servieren.
