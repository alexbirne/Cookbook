\section*{Penne mit Seelachs und Paprika}
\ihead{}\chead{Personen:}\ohead{}
\ifoot{Vorbereitungszeit:}\cfoot{}\ofoot{Kochzeit:\SI{40}{Min}}
{\Large Zutaten}
\begin{multicols}{2}
\begin{itemize}
    \item \SI{270}{g} Penne
    \item \SI{250}{g} Seelachs
    \item \SI{25}{g} Tomatenpesto
    \item \SI{200}{g} Kochsahne
    \item \SI{40}{g} geriebener Hartkäse
    \item \num{1} rote Spitzpaprika
    \item \num{1} Knoblauchzehe
    \item Öl, Wasser, Gemüsebrühe
    \item Salz, Pfeffer, Paprika
\end{itemize}
\end{multicols}
\noindent
{\Large Zubereitung}\\
\\
\textit{Vorbereitung:} Ofen auf \SI{180}{\celsius} Umluft vorheizen.\\
Paprika halbieren, Kerngehäuse entfernen und in etwa \SI{3}{cm} große Stücken schneiden. 
Mit \num{1} EL Öl, Salz und Pfeffer vermengen und auf ein mit Backpapier belegtes Backblech legen und für etwa \SI{15}{min} backen.
Auf einem zweiten mit Backpapier blegten Backblech die Hälfte des geriebenen Käse zu zwei flachen Häufchen verteilen und unter der Paprika für \SIrange{5}{7}{min} mitbacken.
Penne nach Packungsangaben kochen. 
Knoblauch abziehen und fein hacken und Seelach in \SI{3}{cm} große Stücke schneiden.
Sahne, Tomatenpesto, \SI{50}{ml} Wasser, Paprikagewürz, Gemüsebrühe, Salz und Pfeffer vermischen. 
Knoblauch und Fisch in \num{1} EL Öl \SIrange{2}{3}{min} anbraten und mit der vorbereiteten Sahnemischung ablöschen und eindicken lassen.
Den restlichen Käse unterrühren und mit Salz und Pfeffer abschmecken. 
Penne und Sauce mischen, mit Paprikastücken toppen und mit den Käsechips servieren. 
