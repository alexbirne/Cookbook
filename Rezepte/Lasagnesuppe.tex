\section*{Lasagnesuppe}
\ihead{}\chead{Personen: 5}\ohead{}
\ifoot{Vorbereitungszeit:20}\cfoot{}\ofoot{Kochzeit:15}
{\Large Zutaten}
\begin{multicols}{2}
\begin{itemize}
    \item \SI{200}{g} Linsen
    \item \num{1} Zwiebel 
    \item \num{3} Knoblauchzehen
    \item \SI{10}{g} Olivenöl
    \item \SI{400}{g} Hackfleisch, gemischt
    \item \SI{1}{l} Gemüsebrühe
    \item \SI{800}{g} Tomaten, gehackt
    \item \SI{40}{g} Tomatenmark
    \item \SI{200}{g} Lasagneplatten
    \item \SI{150}{g} Creme fraiche
    \item Gouda, Salz, Pfeffer, Kräuter
\end{itemize}
\end{multicols}
\noindent
{\Large Zubereitung}\\
\\
Zuerst die Zwiebeln und den Knoblauch klein würfeln.
Das Olivenöl in einem großen Topf auf mittlerer Hitze erhitzen und die Zwiebeln unter gelegentlichem Rühren glasig andünsten.
Das Hackfleisch hinzugeben und gründlich zerkleinern, damit keine großen Stücke mehr vorhanden sind.
Den Knoblauch hinzugeben und mit Salz, Pfeffer und den Kräutern würzen.
Ist das Hackfleisch scharf angebraten und hat einige Röstaromen, gebt ihr die Brühe, die gehackten Tomaten und das Tomatenmark hinzu und lasst es bei geschlossenem Deckel und kleiner Hitze weiter köcheln.
Die Lasagne-Platten in mundgerechte Stücke brechen, dem Topf hinzufügen und so lange weiter köcheln, bis die Lasagne-Blätter gar sind.
Noch einmal abschmecken und in Schalen portionieren.
Mit etwas geriebenem Käse, einem Klecks Crème fraîche und einigen Kräutern servieren.
