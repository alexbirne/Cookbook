\section*{Spaghetti Carbonara}
\ihead{}\chead{Personen: 2}\ohead{}
\ifoot{Vorbereitungszeit:}\cfoot{}\ofoot{Kochzeit:}
{\Large Zutaten}
\begin{multicols}{2}
\begin{itemize}
    \item \SI{250}{g} Spaghetti (o.ä. Nudeln)
    \item \num{2} Knoblauchzehen
    \item \num{2} Eigelb
    \item \num{2} handvoll gewürfelten Speck
    \item \num{1} Becher Schlagsahne
    \item Parmesan
    \item optional: \num{2} kleine Zwiebeln
    \item optional: kleine getrocknete Chilischoten
    \item Salz, Pfeffer, Muskat
\end{itemize}
\end{multicols}
\noindent
{\Large Zubereitung}\\
\\
Knobluach und Zwiebeln würfeln.
Speck auch, wenn nicht schon fertig gekauft.
Die Chilischoten in kleine Stücke schneiden.
Spaghettiwasser aufsetzen.
Parmesan reiben.
Schlagsahne bereitstellen, in der Spüle ein Sieb vorbereiten.
In einer einigermaßen großen Pfanne etwas Olivenöl erhitzen, Zwiebel, Knoblauch und Speck hineinwerfen.
Auf mittlerer Flamme etwas anbraten, zeitgleich die Spaghetti ins kochende Wasser werfen.
Nach \SIrange{1}{2}{min} erst die Chili-Stücke zum Speck geben, dann die Sahne hinzu leeren.
Die Flamme etwas zurücknehmen, warten bis die Sahne langsam zu kochen beginnt, das Eigelb hinzugeben und unterrühren.
Währenddesssen mit Ofeffer, Muskat und etwas Salz würzen.
Die Sauce vor sich hin broden lassen, ständig rühren, bis sie schön dick geworden ist.
Jetzt ein wenig Parmesan hinzugeben, rühren.
Die Spaghetti sollten inzwischen wunderbar al dente sein.
Abgießen und zur Sauce in die Pfanne werfen.
Auf den Tellern serviert noch etwas Parmesan darüber reiben.