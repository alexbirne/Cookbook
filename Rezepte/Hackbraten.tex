\section*{Hackbraten}
\ihead{}\chead{Personen:4}\ohead{}
\ifoot{Vorbereitungszeit:\SI{40}{Min}}\cfoot{}\ofoot{Kochzeit: \SI{45}{Min}}
{\Large Zutaten}
\begin{multicols}{2}
\begin{itemize}
    \item \num{1.5} Brötchen (altbacken)
    \item \num{2} Gewürzgurken
    \item \num{2} kleine Zwiebeln
    \item \num{1} kleines Bund Petersilie
    \item \num{2} EL Zitronensaft
    \item \SI{50}{g} Butter
    \item \SI{600}{g} gemischtes Hackfleisch
    \item \num{2} Eier
    \item \SI{125}{ml} Fleischbrühe
    \item \SI{125}{ml} Sahne
    \item \num{1} EL Creme fraiche
    \item \num{1} TL Paprikapulver edelsüß
    \item Salz, Pfeffer, Cayennepfeffer
\end{itemize}
\end{multicols}
\noindent
{\Large Zubereitung}\\
\\
\textit{Vorbereitung:} Ofen auf \SI{180}{\celsius} Umluft vorheizen.\\
Die Brötchen in Scheiben schneiden und mit Wasser übergießen.
Etwas quellen lassen und gut ausdrücken. 
Gewürzgurken und Zwiebeln in sehr feine Würfel schneiden.
Ein EL Butter erhitzen und die Zwiebeln glasig anschwitzen. 
Petersilie dazugeben. 
Alles in eine Schüssel geben. 
Brötchen, Gewürzgurken, Hackfleisch, Eier und Zitronensaft hinzufügen. 
Alles mit Salz, Cayennepfeffer und schwarzem PFeffer würzen und durchkneten. 
Die restliche Butter schmelzen. 
Eine Kastenform einfetten.
Den Fleischteig zu einem Laib formen und in die Form legen. 
Auf der unteren Schiene \SI{30}{min} backen. 
Dabei immer mit der flüssigen Butter bestreichen. 
Die Fleischbrühe erhitzen und mit Sahne, Creme fraiche und dem Paprikapulver verrühren. 
Die Sauce über den Hackbraten gießen und weitere \SIrange{10}{15}{min} garen. 
