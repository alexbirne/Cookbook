\section*{Oreo-Erdbeer-Cheesecake}
\ihead{}\chead{Personen:}\ohead{}
\ifoot{Vorbereitungszeit:40}\cfoot{}\ofoot{Kochzeit:}
{\Large Zutaten}
\begin{multicols}{2}
\begin{itemize}
    \item \num{46} Oreokekse
    \item Etwas Butter
    \item \SI{200}{g} Schokolade, weiße
    \item \SI{600}{g} Frischkäse
    \item \num{2} Pck. Vanillezucker
    \item \SI{100}{g} Puderzucker
    \item \SI{200}{g} Erdbeeren
\end{itemize}
\end{multicols}
\noindent
{\Large Zubereitung}\\
\\
\textit{Vorbereitung:} Ofen auf \SI{180}{\celsius} Umluft vorheizen.\\
Für den Boden die Kekse in einer Küchenmaschine sehr gut zerkleinern.
Die Butter in einer Pfanne schmelzen und mit den zerkleinerten Keksen vermischen.
Die Keksmasse in eine Springform geben und am Boden und am Rand festdrücken.
Danach in den Kühlschrank stellen.
Während der Boden auskühlt, die Erdbeeren waschen, putzen und halbieren.
15 Kekse grob hacken.
Wenn der Boden ausgekühlt ist, 150 g der Schokolade klein hacken und mit der Crème fraîche in einem Wasserbad schmelzen lassen.
Frischkäse, Vanillezucker und Puderzucker glatt rühren. 
Die geschmolzene Schokoladencreme etwas abgekühlt unterrühren. 
Ein paar Löffel dieser Schoko-Frischkäse-Masse auf den Tortenboden streichen.
Darauf die halbierten Erdbeeren in einem gewünschten Muster verteilen. 
Die entstehenden Zwischenräume mit den groben Kekskrümeln befüllen.
Die restliche Schoko-Frischkäse-Masse darüber geben und glatt streichen.
Kuchen mit Erdbeeren und restlichen Oreokeksen dekorieren.
Der Kuchen kommt dann für mind. 5 Stunden in den Kühlschrank.
