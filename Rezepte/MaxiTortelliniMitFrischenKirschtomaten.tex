\section*{Maxi-Tortellini mit frischen Kirschtomaten}
\ihead{}\chead{Personen:2}\ohead{}
\ifoot{Vorbereitungszeit:}\cfoot{}\ofoot{Kochzeit:\SI{15}{min}}
{\Large Zutaten}
\begin{multicols}{2}
\begin{itemize}
    \item \SI{400}{g} Maxi-Tortellini mit Ricotta-Spinat Füllung
    \item \SI{200}{g} Sahne
    \item \SI{50}{g} getrocknete Tomaten
    \item \SI{50}{g} Rucola
    \item \SI{12}{ml} Balsamico Creme
    \item \SI{25}{g} Tomatenpesto
    \item \SI{4}{g} Gemüsebrühe
    \item \SI{20}{g} geraspelter Parmesan
    \item \SI{10}{g} Pinienkerne
    \item \SI{250}{g} Kirschtomaten
    \item \SI{15}{ml} Basilikumpaste
    \item \SI{50}{ml} Wasser
    \item Salz, Pfeffer, Olivenöl
\end{itemize}
\end{multicols}
\noindent
{\Large Zubereitung}\\
\\
Einen großen Top mit heißem Wasser füllen, salzen und aufkochen lassen. 
Währenddessen Pinienkerne in einer großen Pfanne ohne Fettzugabe \SIrange{2}{3}{min} fein rösten bis sie duften.
In einer großen Schüssel einen Esslöffen Olivenöl, Rucola, Balsamicocreme, Parmesan, Pinienkerne, Basilikumpaste und getrocknete Tomaten miteinander vermengen und mit Salz und Pfeffer abschmecken.
Pasta in den großen Topf geben.
Hitze reduzieren und Pasta für \SIrange{5}{6}{min} ziehen lassen.
In der Pfanne Sahne, Brühe, Tomatenpesto, Kirschtomaten und Wasser für \SIrange{3}{4}{min} aufkochen lassen bis die Sauce etwas eingedickt ist. 
Pasta durch ein Sieb abgießen und in der Pfanne mit der Sauce mischen. 
Pasta mit Sauce auf tiefen Tellern verteilen, mit dem Rucolasalat toppen und genießen.
