\section*{Gefüllte Paprika}
\ihead{}\chead{Personen: 2}\ohead{}
\ifoot{Vorbereitungszeit:}\cfoot{}\ofoot{Kochzeit:}
{\Large Zutaten}
\begin{multicols}{2}
\begin{itemize}
    \item \SI{500}{g} Hackfleisch
    \item \numrange{1}{2} Zwiebeln
    \item \num{1} Knoblauchzehe
    \item \num{3} Paprikaschoten
    \item \num{1} Ei
    \item \num{1/2} Dose stückige Tomaten (oder frische)
    \item \SI{125}{ml} Gemüsebrühe
    \item \num{1/2} TL Oregano
    \item \num{1/2} TL Basilikum, getrocknet (oder frisch)
    \item \num{1/2} TL Paprika edelsüß
    \item \num{1/4} Tube Tomatenmark
    \item Salz, Pfeffer, Olivenöl
    \item Optional: \num{1} Möhre
\end{itemize}
\end{multicols}
\noindent
{\Large Zubereitung}\\
\\
Zwiebeln und Knoblauch hacken, im Olivenöl glasig dünsten.
Hackfleisch mit Zwiebeln, Ei und den Gewürzen vermengen (evtl. noch die geraspelte Möhre dazu).
Von \num{2} Paprika den Deckel abschneiden und aushöhlen. 
Mit der Fleischmasse füllen und den Rest beiseite legen.
Verbleibende Paprika hacken und mit dem restlichen Hackfleisch in einem großen Top anbraten.
Tomatenmark hinzufügen und mit der Brühe ablöschen.
Die Tomaten hinzugeben.
Die gefüllten Paprikaschoten in den Topf geben und ca \SI{1}{h} bei milder Hitze schmoren.
Dazu Reis, Kartoffeln oder Nudeln reichen.
