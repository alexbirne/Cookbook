\section*{Käsekuchen mit Himbeeren und weißer Schokolade}
\ihead{}\chead{Personen:}\ohead{}
\ifoot{Vorbereitungszeit:}\cfoot{}\ofoot{Kochzeit:\SI{40}{min}}
{\Large Zutaten}
\begin{multicols}{2}
\begin{itemize}
    \item \SI{200}{g} Butterkekse (Vollkorn)
    \item \SI{120}{g} flüssige Butter
    \item \SI{500}{g} Himbeeren
    \item \SI{300}{g} weiße Schokolade
    \item \num{1} Vanilleschote
    \item \SI{500}{g} Frischkäse
    \item \SI{300}{g} Quark
    \item \SI{50}{g} Puderzucker
    \item n.B. weiße Schokospäne zum verzieren
\end{itemize}
\end{multicols}
\noindent
{\Large Zubereitung}\\
\\
Butterkekse in einen Gefrierbeutel geben und zerkleinern.
Die Kekskrümel mit flüssiger Butter mischen, die MAsse in einer \num{26}er Springform verteilen und fest andrücken.
Anschließend im Kühlschrank kalt stellen.
In der Zwischenzeit die weiße Schokolade hacken, über einem Wasserbad schmelzen und lauwarm abkühlen lassen.
Die Vanilleschote der Länge nach aufschneiden und das Mark herausschaben.
Das Vanillemark gemeinsam mit Frischkäse, Quark, Puderzucker und flüssiger Schokolade verquirlen.
Den Keksboden aus dem Kühlschrank nehmen und die Himbeeren darauf verteilen.
Danach die Frischkäsecreme darübergeben und glatt streichen.
Mit Frischhaltefolie abdecken und mindestens \SI{4}{h} (am besten über Nacht) in den Kühlschrank stellen.
Zum Servieren aus der Springform lösen und nach Belieben mit weißen Schokoladenspänen oder einigen Himbeeren verzieren.