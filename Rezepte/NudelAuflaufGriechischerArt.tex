\section*{Nudel-Auflauf nach griechischer Art}
\ihead{}\chead{Personen: 4}\ohead{}
\ifoot{Vorbereitungszeit:}\cfoot{}\ofoot{Kochzeit:\SI{55}{min}}
{\Large Zutaten}
\begin{multicols}{2}
\begin{itemize}
    \item \num{1} gelbe und \num{1} rote Paprika
    \item \num{1} Zwiebel
    \item \num{2} Knoblauchzehen
    \item \num{2} EL Olivenöl
    \item \SI{500}{g} Hackfleisch, gemischt
    \item \num{5} getrocknete Tomaten
    \item \num{1} Dose gehackte Tomaten
    \item \SI{250}{ml} klare Gemüsebrühe
    \item \SI{200}{g} Schlagsahne
    \item \num{1} TL gehackte Petersilie
    \item \SI{250}{g} Kritharaki
    \item \SI{200}{g} Fetakäse
    \item Salz, Pfeffer, Paprika rosenscharf
\end{itemize}
\end{multicols}
\noindent
{\Large Zubereitung}\\
\\
\textit{Vorbereitung:} Ofen auf \SI{160}{\celsius} Umluft vorheizen.\\
Paprika putzen, vierteln und in feine Streifen schneiden.
Zwiebel und Knoblauchzehe abziehen und fein hacken.
Olivenöl in einer großen Pfanne erhitzen.
Das Hackfleisch darin utner Rühren krümelig anbraten.
Nach ca. \SI{5}{min} Zwiebel und Knoblauch zugeben und alles braten, bis es etwas Farbe annimmt.
Mit Salz, Pfeffer, Paprikapulver würzen.
Getrocknete Tomaten in Streifen schneiden und mti den Paprikastreifen unterrühren.
Mit den gehackten Tomaten und Gemüsebrühe auffüllen.
Schlagsahne und Petersilie unterrühren und alles noch einmal gut mit Salz und Pfeffer abschmecken.
Die ungekochten Reisnudeln unterheben.
Alle in eine Auflaufform füllen, den Feta darüberbröseln und im Ofen \SIrange{30}{35}{min} backen.