\section*{Kaiserschmarrn - Tiroler Landgasthofrezept}
\ihead{}\chead{Personen: 4}\ohead{}
\ifoot{Vorbereitungszeit: \SI{30}{Min}}\cfoot{}\ofoot{Kochzeit: \SI{30}{Min}}
{\Large Zutaten}
\begin{multicols}{2}
\begin{itemize}
    \item \SI{100}{g} Rosinen
    \item \num{5} EL Wasser
    \item \num{6} Eigelb
    \item \num{1} Pkg. Vanillezucker
    \item \num{1} gehäufter EL Zucker
    \item \num{1} Prise Salz
    \item \SI{250}{g} Mehl
    \item \SI{500}{ml} Milch
    \item \SI{50}{g} zerlassene Butter
    \item \num{6} Eiweiß
    \item Puderzucker, Butter
\end{itemize}
\end{multicols}
\noindent
{\Large Zubereitung}\\
\\
Die Rosinen im Wasser \SI{0.5}{h} einlegen. 
Die Eigelbe mit dem Vanillezucker, einer Prise Salz und dem Zucker schaumig aufschlagen.
Nach und nach abwechselnd jeweils einen Löffel Mehl und einen guten Schuss Milch einrühren.
Anschließend die zerlassene Butter einrühren. 
Den Teig \SI{30}{min} ruhen lassen, danach nochmals gut durchschlagen. 
Das Eiweiß zu Eischnee schlagen und unter die Teigmasse heben.
Danach die Rosinen unterrühren. 
In eine Pfanne Butter zerlassen und den Teig etwa \SI{1}{cm} hoch eingießen. 
Hitze etwas reduzieren und goldgelb anbacken lassen. 
Die Masse vierteln umdrehen und wieder anbacken lassen.
In Mundgerechte Stücke teilen, mit \num{2} TL Zucker bestreuen und kurz karamelisieren lassen. 
Auf Tellern anrichten und mit Puderzuck bestäuben. 
