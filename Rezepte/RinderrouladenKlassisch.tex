\section*{Rinderrouladen, klassisch}
\ihead{}\chead{Personen:}\ohead{}
\ifoot{Vorbereitungszeit:}\cfoot{}\ofoot{Kochzeit:}
{\Large Zutaten}
\begin{multicols}{2}
\begin{itemize}
    \item \num{8} Rouladen vom Rind
    \item \num{5} Zwiebeln, Halbmonde
    \item \num{12} Scheiben Frühstücksspeck
    \item \num{4} EL Senf
    \item \num{1} Stück Knollensellerie
    \item \num{1} Möhre
    \item \num{1/2} Stange Lauch
    \item \num{1/2} Flasche Rotwein
    \item \SI{1/2}{l} kräftiger Rinderfond
    \item \num{1} TL Speisestärke
    \item Salz, Pfeffer
\end{itemize}
\end{multicols}
\noindent
{\Large Zubereitung}\\
\\
Rouladen waschen und trocken tupfen.
Dann als erstes mit Senf bestreichen, salzen und pfeffern, etwa eine halbe Zwiebel und \num{1.5} Scheiben Frühstücksspeck darauf legen.
Von beiden Längsseiten einschlagen und aufrollen, mit Küchengarn festschnüren.
Rouladen anbraten und in einen Schmortopf legen.
Suppengemüse kleinschneiden und mit der Zwiebeln anbraten.
Jetzt immer wider einen Schuss Rotwein angießen und verkochen lassen, bis der Wein aufgebraucht ist.
Mit der Brühe auffüllen und abschmecken.
Für \SI{1.5}{h} schmoren lassen, evtl. zwischendurch Flüssigkeit angießen.
Sauce durch ein Sieb geben, aufkochen, \num{1} EL Senf und die Speisestärke dazugeben. 
Abschmecken.