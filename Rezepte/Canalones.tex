\section*{Canalones}
\ihead{}\chead{Personen: 6}\ohead{}
\ifoot{Vorbereitungszeit:}\cfoot{}\ofoot{Kochzeit: \SI{55}{min}}
{\Large Zutaten}
\begin{multicols}{2}
\begin{itemize}
    \item \num{1} Zwiebel
    \item \numrange{1}{2} Tomaten
    \item \SI{500}{g} Hackfleisch, gemischt
    \item \numrange{1}{2} EL Mehl
    \item etwa \SI{0.5}{l} Milch 
    \item \numrange{30}{40} Nudelplatten (Cannelloni)
    \item viel Bechamelsauce
    \item \num{1} Pkg geriebener Käse
    \item Olivenöl zum Anbraten
    \item Salz, Pfeffer
\end{itemize}
\end{multicols}
\noindent
{\Large Zubereitung}\\
\\
Zwiebel klein schneiden und in Öl anbraten
Tomaten mit heißem Wasser übergießen und pellen, dann klein schneiden und zu den Zwiebeln in die Pfanne geben.
Nach wenigen Mintuten das Hackfleisch dazugeben und mitbraten.
Alles mit Salz und Pfeffer würzen.
Zunächst das Mehl in die Pfanne geben und verrühren. 
Anschließend soviel Milch in die Pfanne geben, bis eine Masse entsteht, die nicht an der Pfanne klebt.
Das ganze auf großen Tellern ausbreiten und abkühlen lassen.\\
\textit{Vorbereitung:} Ofen auf \SI{180}{\celsius} Umluft vorheizen.\\
Die Nudelplatten in kochendes Wasser geben und nach Anleitung kochen. 
Dabei im Topf regelmäßig, vorsichtig umrühren und nach dem Kochen auf saubere Trockentücher legen.
Die Masse auf die Platten verteilen und aufdrehen, dann in eine rechteckige Auflaufform legen und mit einer Bechamelsauce bedecken und Käse bestreuen.
Im Backofen bei ca. \SI{180}{\celsius} eta \SI{30}{min} backen.

