\section*{Gulasch}
\ihead{}\chead{Personen: 3}\ohead{}
\ifoot{Vorbereitungszeit:}\cfoot{}\ofoot{Kochzeit:}
{\Large Zutaten}
\begin{multicols}{2}
\begin{itemize}
    \item \SI{500}{g} Rindfleisch
    \item \SI{500}{g} Zwiebeln
    \item \num{2} Knoblauchzehen
    \item \num{3} EL Butterschmalz
    \item Salz, Pfeffer, edelsüße Paprika
    \item \num{1/2} EL Weizenmehl
    \item \num{1/2} EL Tomatenmark
    \item \num{1/2} EL Essig
    \item \SI{250}{ml} Rotwein
    \item \SI{500}{ml} Rinderbrühe
    \item \num{1} Lorbeerblatt
    \item \num{1} Bund Suppengrün
    \item \num{1} TL Zitronenabrieb
    \item Speisestärke
\end{itemize}
\end{multicols}
\noindent
{\Large Zubereitung}\\
\\
Zwiebeln, Knoblauch und Fleisch würfeln.
\num{1.5} EL Butterschmalz erhitzen und Fleisch anbraten und kräftig würzen.
Mit Mehl bestäuben und herausnehmen.
Restlichen Butterschmalz erhitzen, Zwiebeln und Knoblauch anschwitzen.
Tomatenmark unterrühren, leicht anrösten, dann mit Essig ablöschen.
Mit Rotwein und BRühe auffüllen. 
Fleisch und Lorbeerblatt zugeben, aufkochen und abgedeckt bei geringer Hitze \SIrange{1.5}{2}{h} schmoren.
Nach \SI{45}{min} das SUppengrün zum Fleisch geben, nach Ende der Kochzeit das Lorbeerblatt entfernen.
Mit Zitrone und Speisestärke zur gewünschten Konsistenz aufkochen.
Dazu Spätzle.