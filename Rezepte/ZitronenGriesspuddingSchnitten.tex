\section*{Zitronen-Grießpudding-Schnitten}
\ihead{}\chead{Personen:}\ohead{}
\ifoot{Vorbereitungszeit:}\cfoot{}\ofoot{Kochzeit:}
{\Large Zutaten}
\begin{multicols}{2}
\textit{Boden:}
\begin{itemize}
    \item \SI{150}{g} Butter
    \item \SI{260}{g} Mehl
    \item \num{1/2} Pkg Backpulver
    \item \SI{80}{g} Puderzucker
    \item \num{1} Ei
    \item Salz
\end{itemize}
\textit{Creme:}
\begin{itemize}
    \item \SI{500}{g} Creme fraiche
    \item \SI{90}{g} Zucker
    \item \num{3} Eier
    \item \num{1} Zitrone (Saft und Schale)
    \item \num{1} Pkg Puddingpulver für Grießpudding
\end{itemize}
\textit{Belag:}
\begin{itemize}
    \item \SI{500}{ml} Milch 
    \item \num{1} Pkg Zitronenpuddingpulver
    \item \num{2} EL Zucker
\end{itemize}
\end{multicols}
\noindent
{\Large Zubereitung}\\
\\
\textit{Vorbereitung:} 
\begin{itemize}
    \item \num{30}x\num{20}\,\si{cm} Backrahmen benötigt
    \item Ofen auf \SI{160}{\celsius} Umluft vorheizen.
\end{itemize}
Mürbeteig zubereiten und ca. \SI{1/2}{h} in den Kühlschrank legen.
Eine Form mit dem Teig auslegen und im vorgehizten Backofen bei \SI{160}{\celsius} \SI{12}{min} backen.
Für die Füllung Eier mit Zucker schaumig schlagen, Creme fraiche, Zitrone und Grießpuddingpulver verrühren und auf den Kuchen aufstreichen. 
Bei \SI{140}{\celsius} noch etwa \SI{30}{min} weiterbacken.
Zitronenpudding bereiten und aufstreichen und erkalten lassen.
Evtl. mit Zitronenglasur bespritzen.