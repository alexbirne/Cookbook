\section*{Red Velvet Cheesecake Brownies}
\ihead{}\chead{Personen:}\ohead{}
\ifoot{Vorbereitungszeit:}\cfoot{}\ofoot{Kochzeit:}
{\Large Zutaten}
\begin{multicols}{2}
\textit{Red Velvet Schicht:}
\begin{itemize}
    \item \SI{200}{g} Mehl
    \item \SI{250}{g} Zucker
    \item \num{1.5} EL ungesüßtes Kakaopulver
    \item \num{0.5} TL Backpulver
    \item \num{1} Prise Salze
    \item \SI{120}{ml} Pflanzenöl
    \item \num{2} Eier
    \item \num{1} TL Weißweinessig
    \item \num{1} TL Vanilleextrakt
    \item Rote Lebensmittelfarbe
    \item \num{4} EL Milch (\num{2} für den Boden, \num{2} für die obere Schicht)
\end{itemize}
\textit{Cheesecake-Schicht:}
\begin{itemize}
    \item \SI{230}{g} Frischkäse
    \item \SI{80}{g} Zucker
    \item \num{1} Ei
    \item \num{1} TL Vanilleextrakt
\end{itemize}
\end{multicols}
\noindent
{\Large Zubereitung}\\
\\
\textit{Vorbereitung:} 
\begin{itemize}
    \item \num{26}x\num{20}\,\si{cm} Backrahmen benötigt
    \item Ofen auf \SI{180}{\celsius} Umluft vorheizen.
\end{itemize}
Die trockenen Zutaten vermischen, dann die flüssigen nach und nach hinzufügen (außer die letzten \num{2} EL Milch) und vermengen.
Die Hälfte vom Teig abnehmen und in den Backrahmen streichen. 
Die restliche Milch zum Teigrest hinzufügen und beiseite stellen.
Jetzt alle Zutaten der Cheesecake-Schicht verrühren und auf den Teigboden geben.
Danach den restlichen roten Teig in Kleksen daraufgeben und mit einem Messer marmorieren. 
Bei \SI{180}{\celsius} Umluft \SIrange{20}{25}{min} backen.
