\section*{Gebackene Laugen-PetersilienKnödel}
\ihead{}\chead{Personen:2}\ohead{}
\ifoot{Vorbereitungszeit:}\cfoot{}\ofoot{Kochzeit:\SIrange{40}{50}{min}}
{\Large Zutaten}
\begin{multicols}{2}
\begin{itemize}
    \item \SI{270}{g} Laugenstangen (etwa \num{3})
    \item \num{2} Stangen Porree
    \item \SI{200}{g} Sahne
    \item \SI{200}{ml} Milch
    \item \SI{100}{g} Hirtenkäse
    \item \SI{20}{g} Petersilie
    \item Salz, Pfeffer, Olivenöl, Muskat
\end{itemize}
\end{multicols}
\noindent
{\Large Zubereitung}\\
\\
\textit{Vorbereitung:} Ofen auf \SI{200}{\celsius} Umluft vorheizen.\\
Zu Beginn: Milch in einem Topf erwärmen. 
Laugenstangen in \SI{1}{cm} große Würfel schneiden. 

Laufenmassen vorbereiten: Laugenstangenwürfel mit warmer Milch übergießen. 
Mit Salz und Pfeffer würzen und etwas abkühlen lassen. 

Porree schneiden: Porree in feine Streifen schneiden. 
Petersilie fein hacken. 
Hirtenkäse mit den Händen zerbröseln. 

Laugenknödel backen: Petersilie und die Hälfte des Hirtenkäses mit der Laugenmasse vermischen.
\num{6} Knödel formen und in eine leicht gefette Auflaufform geben. 
Restlichen Hirtenkäse über die Knödel geben.
\SIrange{15}{20}{min} backen.

Gemüse zubereiten: In einer Pfanne ein EL Öl erhitzen und Porree darin \SIrange{6}{7}{min} braten. 
Mit Sahne, \SI{50}{ml} Wasser und Gewürzen (inkl. Muskat) ablöschen. 
\SIrange{3}{4}{min} köcheln lassen. 
Mit Salz und Pfeffer abschmecken. 

Anrichten: Porree auf Tellern verteilen. 
Je \num{3} Knödel darauf anrichten und genießen. 