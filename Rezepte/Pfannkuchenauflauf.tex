\section*{Maxi-Tortellini mit frischen Kirschtomaten}
\ihead{}\chead{Personen:2}\ohead{}
\ifoot{Vorbereitungszeit:}\cfoot{}\ofoot{Kochzeit:\SIrange{45}{55}{min}}
{\Large Zutaten}
\begin{multicols}{2}
\begin{itemize}
    \item \SI{100}{g} Babyspinat
    \item \SI{100}{g} Frischkäse
    \item \SI{100}{g} Weizenmehl
    \item \SI{400}{g} stückige Tomaten mit Basilikum
    \item \SI{40}{g} geriebener Parmesan
    \item \SI{100}{g} Hirtenkäse
    \item \SI{10}{g} Mandelstifte
    \item \SI{4}{g} Gemüsebrühe 
    \item \SI{200}{ml} Milch
    \item \num{1} Zwiebel
    \item \num{1} Knoblauchzehe
    \item \num{1} Ei
    \item Salz, Pfeffer, Olivenöl, Gewürze
\end{itemize}
\end{multicols}
\noindent
{\Large Zubereitung}\\
\\
\textit{Vorbereitung:} Ofen auf \SI{200}{\celsius} Umluft vorheizen.\\
Pfannkuchenteig zubereiten: In einer großen Schüssel die Milch, etwas Salz, und ein Ei gut verrühren. 
Mehl nach und nach hinzugeben und glatt rühren. 
Den Teig jetzt etwas ruhen lassen.

Für die Füllung: Babyspinat fein hacken. In einer weiteren großen Schüssel Babyspinat mit etwas Frischkäse, Mandelstiften und Parmesan gut vermischen und mit Salz und Pfeffer abschmecken.

Gemüse schneiden: Zwiebel und Knoblauch abziehen. Zwiebel halbieren und fein würfeln. 
Knoblauch fein hacken. 

Pfannkuchen backen: Aus dem Teig \num{6} Pfannkuchen backen. 
Dazu in einer beschichteten Pfanne etwa \num{1} Esslöffel Öl bei mittlerer Hitze erwärmen. 
Etwas Teig in die Pfanne geben und je Seite etwa \SI{2}{min} anbraten und wenden.
Zwischendurch einige Tropfen Öl nachgießen.
Pfannkuchen etwas abkühlen lassen.

Tomatensauce kochen: In der Pfanne erneut einen Esslöffel Öl erwärmen. 
Zwiebeln und Knoblauch darin \SIrange{2}{3}{min} anbraten, gehackte Tomaten, Gemüsebrühpulver und Gewürze dazugeben und \SI{1}{min} köcheln lassen.

Auflauf vollenden: Alle Pfannkuchen gleichmäßig mit etwas von der Frischkäsemischung füllen und zusammenrollen. 
Röllchen eng nebeneinander in eine Auflaufform legen.
Sauce darübergießen und Hirtenkäse mit den Händen darüberbröseln. 
Alles etwa \SI{10}{15}{min} im Ofen backen. 
Auflauf aus dem Ofen nehmen, auf Teller verteilen und genießen. 
