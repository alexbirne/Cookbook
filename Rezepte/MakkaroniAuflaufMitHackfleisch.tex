\section*{Makkaroni-Auflauf mit Hackfleisch, auf die leichte Tour}
\ihead{}\chead{Personen: 4}\ohead{}
\ifoot{Vorbereitungszeit:}\cfoot{}\ofoot{Kochzeit:\SI{25}{min}}
{\Large Zutaten}
\begin{multicols}{2}
\begin{itemize}
    \item \SI{500}{g} Makkaroni
    \item \SI{300}{g} Hackfleisch vom Rind
    \item \num{1} große Zwiebel
    \item \num{1} Dose Tomatensuppe
    \item \num{1/2} Pkg geriebener Käse
    \item Margarine
    \item Olivenöl
    \item Salz, Pfeffer, Paprikapulver edelsüß
    \item optional: Creme fine
\end{itemize}
\end{multicols}
\noindent
{\Large Zubereitung}\\
\\
\textit{Vorbereitung:} Ofen auf \SI{200}{\celsius} Umluft vorheizen.\\
Makkaroni in Salzwasser al dente kochen.
In der Zwischenzeit das Hackfleisch in einer Pfanne in etwas Olivenöl scharf anbraten.
Die Zwiebel fein würfeln und zu dem Hackfleisch geben.
Kurz zusammen weiterbraten, bis die Zwiebeln glasig sind.
Mit Salz, Pfeffer und Paprika würzen.
Darf ruhig schön würzig sein.
Mit der Tomatensuppe ablöschen und aufkochen lassen.
Mit Cremefine verfeinern und gegebenfalls nachwürzen.
Makkaroni abschrecken und mit der Hackfleischsauce vermischen.
Das ganze in einer Auflaufform verteilen, den Käse darüber streuen und bei ca. \SI{200}{\celsius} so lange backen, bis der Käse zerlaufen ist, bzw. die gewünschte Bräunung erreicht hat.