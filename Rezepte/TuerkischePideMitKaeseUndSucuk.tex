\section*{Türkische Pide mit Käse und Sucuk}
\ihead{}\chead{Personen:4}\ohead{}
\ifoot{Vorbereitungszeit:\SI{65}{min}}\cfoot{}\ofoot{Kochzeit:\SI{55}{min}}
{\Large Zutaten}
\begin{multicols}{2}
\textit{Teig:}
\begin{itemize}
    \item \SI{600}{g} Mehl
    \item \num{1} Pkg Trockenhefe
    \item \SI{250}{ml} lauwarmes Wasser
    \item \num{4} EL Joghurt
    \item \num{4} EL Olivenöl
    \item \num{1.5} TL Salz
    \item \num{1} TL Zucker
\end{itemize}
\textit{Füllung:}
\begin{itemize}
    \item \SI{400}{g} geriebener Käse (Kasar)
    \item \num{20} Scheiben Sucuk
\end{itemize}
\textit{Außerdem:}
\begin{itemize}
    \item \num{1} Ei
    \item Mehl für die Arbeitsfläche
\end{itemize}
\end{multicols}
\noindent
{\Large Zubereitung}\\
\\
\textit{Vorbereitung:} Ofen auf \SI{200}{\celsius} Ober-/Unterhitze vorheizen.\\
Für den Teig die Hefe mit dem Wasser verrühren und \SI{5}{min} stehen lassen. 
Danach zum Mehl geben. 
Gemeinsam mit Joghurt, Salz, Zucker und Öl in etwa \SI{7}{min} geschmeidig kneten. 
Eine Schüssel mit Öl ausfetten und den Teig hineingeben. 
Die Oberfläche mit Öl benetzen. 
Den Teig bei Zimmertemperatur \SI{60}{min} gehen lassen. 
Den Teig anschließend in zwei oder vier Teile schneiden und jeweils zu einer Kugel kneten. 
Auf einer bemehlten Arbeitsfläche jeden Kugel Oval ausrollen.
Die fertigen Stücke auf ein Backblech legen. 
Reichlich mit Käse belegen und zu Schiffchen formen. 
Auf den Käse kommen noch einige Scheiben Sucuk. 
Die Ränder des Teiges mit dem verquirlten Ei bestreichen. 
Dann kommt alles für etwa \SI{15}{min} in den Backofen. 
