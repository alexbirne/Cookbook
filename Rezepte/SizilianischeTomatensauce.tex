\section*{Pasta mit sizilianischer Tomatensauce}
\ihead{}\chead{Personen: 4}\ohead{}
\ifoot{Arbeitszeit: 20 Min}\cfoot{}\ofoot{Kochzeit:2h 30 Min}
{\Large Zutaten}
\begin{multicols}{2}
\begin{itemize}
    \item \num{10} große Tomaten oder 2 große Dosen Tomaten
    \item \num{1} Zwiebel
    \item \num{1} Knoblauchzehe
    \item \num{1} Prise Zucker
    \item \num{1} EL Olivenöl
    \item \num{1} EL Salz
    \item \num{1} Prise Pfeffer
    \item \num{3} Basilikumblätter
\end{itemize}
% \columnbreak
\end{multicols}
\noindent
{\Large Zubereitung}\\
\\
\textit{Vorbereitung:} Ofen auf \SI{180}{\celsius} Umluft vorheizen.\\
Die frischen Tomaten waschen, in Würfel schneiden und entkernen, oder Tomaten aus der Konserve nehmen.
Die Zwiebeln schälen und in Würfel schneiden.
Den Knoblauch abziehen und pressen. 
Das Olivenöl erhitzen, nicht zu heiß, sonst werden die Zwiebeln schnell dunkel.
Erst die Zwiebeln darin anschwitzen, damit sie glasig werden.
Die Tomaten dazugeben und mit etwas Wasser ablöschen.
Dann Zucker, Salz und Pfeffer dazugeben.
Etwas zugedeckt etwa 20 Min. sanft köcheln lassen, danach mit einem Pürierstab zu einer feinen Soße zerkleinern oder durch ein feinmaschiges Sieb drücken bzw. mit einer Flotten Lotte durchsieben.
Die ganze durchgesiebte Soße zurück auf den Herd geben.
Den Knoblauch dazugeben und unter Beobachtung etwa 1 - 2 Stunden bei niedriger Wärmezufuhr köcheln lassen.
Wenn es zu dick wird, mit ein wenig Wasser wieder flüssiger machen.
Mit Salz, Pfeffer, Basilikum abschmecken.
Dazu Pasta nach Wahl.
