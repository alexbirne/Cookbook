\section*{Maxi-Ravioli mit Tomate-Mozzarella-Füllung}
\ihead{}\chead{Personen:2}\ohead{}
\ifoot{Vorbereitungszeit:}\cfoot{}\ofoot{Kochzeit:\SIrange{25}{35}{min}}
{\Large Zutaten}
\begin{multicols}{2}
\begin{itemize}
    \item \SI{400}{g} Maxi-Ravioli Tomate-Mozzarella
    \item \num{1} Knoblauchzehe
    \item \SI{200}{g} Sahne
    \item \SI{20}{g} geriebener Hartkäse
    \item \num{1} Zucchini
    \item \SI{50}{ml} Gemüsebrühe
    \item \SI{125}{g} Kirschtomaten
    \item \SI{15}{ml} Basilikumpaste
    \item Salz, Pfeffer, Olivenöl, Wasser
\end{itemize}
\end{multicols}
\noindent
{\Large Zubereitung}\\
\\
\textit{Vorbereitung:} Ofen auf \SI{200}{\celsius} Umluft vorheizen.\\
Zu Beginn: Enden der Zucchini abschneiden.
Zucchini längs halbieren und in \SI{0.5}{cm} dünne Halbmonde schneiden. 

Ofengemüse backen: Zucchini und Kirschtomaten auf einem mit Backpapier belegten Backblech verteilen, mit \num{1} EL Öl, Pfeffer und Salz vermengen. 
Gemüse für etwa \SI{15}{min} backen, bis es weich ist und anfängt zu bräunen. 

Pasta kochen: Einen großen Topf mit Wasser füllen, salzen, Ravioli hineingeben und \SIrange{3}{4}{min} bissfest gar ziehen lassen.
Ravioli durch ein Sieb abgießen und abtropfen lassen.
Knoblauchzehe abziehen.  

Sauce zubereiten: Den Top auswischen und die Gemüsebrühe und Kochsahne hineingeben und zum kochen bringen.
Knoblauch dazupressen und \SIrange{2}{3}{min} köcheln lassen, bis die Sauce ein bisschen eingedickt ist. 
Mit Salz und Pfeffer abschmecken. 

Pasta vollenden: Ravili zu der Sauce geben und alles vorsichtig vermengen. 
Am Ende der Backzeit Basilikumpaste mit dem Ofengemüse auf dem Backblech vermengen. 

Anrichten: Ravioli auf Tellern verteilen und Ofengemüse darauf anrichten. Pasta mit Hartkäse toppen und genießen.  