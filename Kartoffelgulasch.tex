\section*{Kartoffelgulasch}
\ihead{}\chead{Personen: 3}\ohead{}
\ifoot{Vorbereitungszeit:}\cfoot{}\ofoot{Kochzeit: 30 Min}
{\Large Zutaten}
\begin{multicols}{2}
\begin{itemize}
    \item \SI{750}{g} Kartoffeln
    \item \num{3} Zwiebeln, gehackt
    \item \SI{200}{g} Speck, gewürfelt
    \item \num{3} EL Paprikapulver, mild
    \item \num{1} TL Chilipulver
    \item \num{1} Liter Fleisch- oder Gemüsebrühe
    \item \num{1/2} EL Tomatenmark
    \item \num{1} Bund Schnittlauch, gehackt
    \item \num{1} EL Petersilie
    \item \SI{125}{ml} Rotwein, trocken
    \item \num{1} Prise Zucker
    \item \num{1} Prise Muskat
    \item \SI{50}{g} Butter
    \item \num{1} EL Olivenöl
    \item Salz, Pfeffer
\end{itemize}
% \columnbreak
\end{multicols}
\noindent
{\Large Zubereitung}\\
\\
Kartoffeln schälen und in daumengroße Stücke zerteilen. 
Die Butter und das Olivenöl in einem großen Topf auf großer Stufe erhitzen. 
Die Kartoffeln etwas trockentupfen und in dem Topf anbraten, so dass sie leicht braun von allen Seiten sind. 
Die Zwiebeln und den Speck dazugeben und goldgelb mitbraten. 
Das Paprikapulver über den Kartoffeln verteilen und alles schön vermengen. 
Alles schön heiß werden lassen und mit dem Rotwein ablöschen.
Die Brühe dazugeben und einmal aufkochen lassen, dann auf niedrigste Stufe zurückstellen, so dass es leicht köchelt. 
Tomatenmark, den Schnittlauch, die Knoblauchzehe (ganz), die Petersilie und das Chilipulver, Salz, Pfeffer und Zucker dazugeben. 
Alles schön langsam etwa 1 Stunde bei geschlossenem Topf köcheln lassen. 
Zum Schluss mit dem Muskat verfeinern und in tiefen Tellern anrichten.


